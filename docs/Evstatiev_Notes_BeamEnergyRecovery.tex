\documentclass[11pt]{article}

\usepackage{amsmath}
\usepackage[left=0.5in,right=0.5in,top=0.5in,bottom=0.75in]{geometry}
\usepackage{color}
\usepackage{graphicx}

\newcommand{\LL}{\mathcal{L}}
\newcommand{\HH}{\mathcal{H}}

\date{\today} %{10 June 2016}
\author{E.~G. Evstatiev}
\title{Notes on beam energy recovery model}

\begin{document}

\maketitle

\section{General case}
\label{General_case}

This calculation repeats a major part of the calculation from my paper in POP~\cite{evstatiev_relativistic_2005}. 
The system of beam particles and fields is described by the following Lagrangian:
%
\begin{align}
\LL =& \sum_{j=1}^{N_p}\left\{ -mc^2\sqrt{1-\frac{\dot{\mathbf{r}}_j} {c^2}} 
-\frac{e}{c}\dot{\mathbf{r}}_j \cdot \mathbf{A}(\mathbf{r}_j) \right\}
+\frac{1}{8\pi}\int \!\!d^3 x \left(\left|\frac{1}{c}\frac{\partial \mathbf{A}(\mathbf{x})}{\partial t}\right|^2
 - \left| \nabla\times \mathbf{A}(\mathbf{x}) \right|^2  \right)
\label{L_cont}
\end{align}
%
For now, let's neglect the electrostatic potential (beam self-interaction) and concentrate on the 
exchange of energy between the beam and the RF mode(s).

We represent the magnetic vector potential as:
%
\begin{equation}
\mathbf{A}(\mathbf{x}) = \sum_{l>0}\left[ \mathbf{a}_l(t)\psi_l(\mathbf{x}) + \mathbf{a}_l(t)^*\psi_l^*(\mathbf{x}) 
\right]
\label{A_repr}
\end{equation}
%
where $\psi_l$ are the cavity eigenmodes with index $l=(l_1,l_2,l_e)$ and the amplitudes $\mathbf{a}_l(t)$ include
a fast time scale $e^{-i\omega_l t}$ and a slow time scale describing the (time-averaged) energy exchange between
fields and particles. 
%
We use the adiabatic approximation
%
\begin{equation}
\left| \dot{\mathbf{a}}_l(t)\right| \ll \omega_l \left| \mathbf{a}_l(t)\right|
\label{Adiab_assumpt} 
\end{equation}
%
and then the time derivative of this expression becomes
\begin{equation}
\dot{\mathbf{a}}_l(t) = -i\omega_l \mathbf{a} + \underline{\dot{\mathbf{a}}}_l\, .
\label{Adiab_time_deriv} 
\end{equation}
%
The underline denotes the time derivative with respect to the slow time scale only; this notation is auxiliary
and not used in the final result, but it is convenient.
%
We assume the eigenfunctions are normalized as:
%
\begin{equation}
\int \!\!d^3 x\,  \psi_l(\mathbf{x}) \psi_{l'}^*(\mathbf{x}) = \delta_{l l'}\, .
\label{Eigenfunction_norm} 
\end{equation}
%

The goal is now to use a finite number of eigenmodes in the Lagrangian, so we substitute \eqref{A_repr} into
\eqref{L_cont}. Consider the two field terms separately:
%
\begin{align}
\frac{1}{8\pi c^2}\int \!\!d^3 x\,\left|\frac{\partial \mathbf{A}(\mathbf{x})}{\partial t}\right|^2 =&
\frac{1}{8\pi c^2}\int \!\!d^3 x\, \left|  \sum_{l>0} 
\left[-i\omega_l \left(\mathbf{a}_l(t)\psi_l(\mathbf{x}) - \mathbf{a}_l^*(t)\psi_l^*(\mathbf{x}) \right)
+ \left(\underline{\dot{\mathbf{a}}}_l(t)\psi_l(\mathbf{x}) 
+ \underline{\dot{\mathbf{a}}}_l^*(t)\psi_l^*(\mathbf{x}) \right)
 \right] \right|^2 \nonumber \\
\simeq & \frac{1}{4\pi c^2}\sum_{l>0}\left[ \omega_l^2|\mathbf{a}_l|^2 
- i\omega_l( \mathbf{a}_l\cdot\underline{\dot{\mathbf{a}}}_l^* - \mathbf{a}^*\cdot\underline{\dot{\mathbf{a}}}_l ) 
\right]
\label{A_dt_term}
\end{align}
%
and
%
\begin{align}
-\frac{1}{8\pi} \int \!\!d^3 x\,\left| \nabla\times \mathbf{A} \right|^2 =&
-\frac{1}{8\pi} \int \!\!d^3 x\,\left| \sum_{l>0}\left( \nabla\psi_l\times \mathbf{a}_l 
+ \nabla{\psi^*}\times\mathbf{a}^*\right) \right|^2 \nonumber \\ 
\simeq& - \frac{1}{4\pi}\sum_{l>0}\gamma_l^2|\mathbf{a}_l|^2
\label{A_curl_term}
\end{align}
%
In the last calculation we have used $\mathbf{a}_l\cdot\nabla\psi_l = 0$ (following from $\nabla\cdot\mathbf{A}=0$).
We also used
%
\begin{equation}
\int \!\!d^3 x\, \nabla\psi_l\cdot\nabla\psi_{l'}^* = 
- \int \!\!d^3 x\, \psi_l \nabla^2\psi_{l'} = 
\gamma_l^2\delta_{ll'}
\label{Using_int_nabla_psi_squared}
\end{equation}
%
with eigenvalue $\gamma_l^2$
%
\begin{equation}
\nabla^2\psi_l = -\gamma_l^2\psi_l\, .
\label{Eigenvalue}
\end{equation}
%
We have neglected quadratic terms in the slow time scale derivative, i.e., quadratic in 
$\underline{\dot{\mathbf{a}}}_l$.

Adding the two terms \eqref{A_dt_term} and \eqref{A_curl_term} we have:
%
\begin{align}
\frac{1}{8\pi c^2}\int \!\!d^3 x\,\left|\frac{\partial \mathbf{A}(\mathbf{x})}{\partial t}\right|^2
-\frac{1}{8\pi} \int \!\!d^3 x\,\left| \nabla\times \mathbf{A} \right|^2 \simeq& 
\frac{1}{4\pi}\sum_{l>0}\left( \frac{\omega_l^2}{c^2} - \gamma_l^2 \right) |\mathbf{a}_l|^2
-\frac{1}{4\pi}\sum_{l>0}\frac{i\omega_l}{c^2}\left(  \mathbf{a}_l\cdot\underline{\dot{\mathbf{a}}}_l^* - \mathbf{a}^*\cdot\underline{\dot{\mathbf{a}}}_l  \right) \nonumber\\
=& -\frac{1}{2\pi} \sum_{l>0}\frac{\omega_l^2}{c^2}\left( \mathbf{a}_l\cdot\mathbf{a}^*\right) 
+ \frac{1}{4i\pi}\sum_{l>0}\frac{\omega_l}{c^2}\left(  \mathbf{a}_l\cdot{\dot{\mathbf{a}}}_l^* - \mathbf{a}^*\cdot{\dot{\mathbf{a}}}_l \right)
\label{A_averaged}
\end{align}
%
where we have used the dispersion relation for the cavity $\omega_l^2 = c^2\gamma_l^2$ 
to eliminate the first term in the first line of Eq.~\eqref{A_averaged} and formula \eqref{Adiab_time_deriv} 
to express $\underline{\dot{\mathbf{a}}}_l$ through $\mathbf{a}_l$ and ${\dot{\mathbf{a}}}_l$.

Then we have a Lagrangian for the combined system of fields and particles:
%
\begin{align}
\LL =&  -mc^2\sum_{j=1}^{N_p}\sqrt{1-\frac{\dot{\mathbf{r}}_j} {c^2}} 
-\frac{e}{c}\sum_{j=1}^{N_p}\sum_{l>0}\dot{\mathbf{r}}_j \cdot 
\left[ \mathbf{a}_l(t)\psi_l(\mathbf{r}_j) + \mathbf{a}_l(t)^*\psi_l^*(\mathbf{r}_j) \right] \nonumber\\
& - \frac{1}{2\pi} \sum_{l>0}\frac{\omega_l^2}{c^2}\left( \mathbf{a}_l\cdot\mathbf{a}_l^*\right) 
+ \frac{1}{4i\pi}\sum_{l>0}\frac{\omega_l}{c^2}\left(  \mathbf{a}_l\cdot{\dot{\mathbf{a}}}_l^* - \mathbf{a}_l^*\cdot{\dot{\mathbf{a}}}_l \right).
\label{L_reduced}
\end{align}
%

Only a finite number of modes are to be used in \eqref{L_reduced}.

\section{Rectangular cavity}
\label{Rectangular cavity}

Consider a rectangular cavity of $x$-, $y$-, $z$-dimensions $(a,b,d)$ and volume $V=abd$.

\subsection{TM modes}
\label{TM_rectangular_modes}

For TM modes the normalized eigenfunction $\psi_l$ (with $l=(m,n,p)$\,) is:
%
\begin{equation}
\psi_{mnp}(x,y,z) = C_{mnp}\sin\left(\frac{m\pi x}{a}\right) \sin\left(\frac{n\pi y}{b}\right) 
\cos\left(\frac{p\pi z}{d}\right), \quad 
C_{mnp} = \left\{
\begin{tabular}{l}
$\frac{2}{\sqrt{V}}, \quad p = 0$ , \\
$\frac{2\sqrt{2}}{\sqrt{V}}, \quad p\ne 0$ .
\end{tabular}
\right.
\label{psi_rectangular_TM}
\end{equation}
%
the eigenvalue is
%
\begin{equation}
\nabla^2\psi_{mnp} = -\gamma_{mnp}^2\psi_{mnp}, \qquad
\gamma_l^2\equiv \gamma_{mnp}^2 = \left(\frac{m\pi}{a} \right)^2 + \left(\frac{n\pi}{b} \right)^2
+ \left(\frac{p\pi}{d} \right)^2
\label{eigenvakue_rectangular_TM}
\end{equation}
%

\subsection{Lagrangian for the special case of rectangular modes}
\label{L_rectangular_TM}

For a one-dimensional calculation we can set $x=a/2$ and $y=b/2$ (axis of the cavity) with the lowest order mode
 TM$_{110}$. Assuming the $x$- and $y$-dependence of the fields doesn't change, we can reduce the Lagrangian
\eqref{L_reduced} to only $z$-dependent quantities:
%
\begin{align}
\LL =& -mc^2\sum_{j=1}^{N_p}\sqrt{1-\frac{\dot{{z}}_j} {c^2}} 
- \frac{e}{c} \left(\frac{2}{\sqrt{V}}\right)
\sum_{j=1}^{N_p} \dot{z}_j \left[a_z(t) + a_z^*(t)  \right] \nonumber\\
& - \frac{1}{2\pi} \frac{\omega_{_{110}}^2}{c^2}\left( {a}_z {a}_z^*\right) 
+ \frac{1}{4i\pi} \frac{\omega_{_{110}}}{c^2}\left( {a}_z {\dot{{a}}}_z^* - {a}_z^*{\dot{{a}}}_z \right)
\label{L_z}
\end{align}
%
where
%
\begin{equation}
\omega_{_{110}}^2 = \pi^2 c^2 \left( \frac{1}{a^2} + \frac{1}{b^2} \right), \qquad
\psi_{_{110}}(z) = C_{110} = \frac{2}{\sqrt{V}} = \mbox{const}.
\end{equation}

\subsection{Equations of motion for particles and fields}
\label{EOM}

The equations of motion become as follows. For the particles we have:
%
\begin{equation}
\frac{d}{dt}\frac{\partial\LL}{\partial \dot{z}_j} = \frac{\partial \LL}{\partial z_j}, \qquad j=1,2,\ldots, N_p \nonumber
\end{equation}
%
which gives
%
\begin{equation}
\ddot{z}_j = \frac{e}{mc}\frac{2}{\sqrt{V}}\left(1-\frac{\dot{z}_j^2}{c^2} \right)^{3/2} (a_z + a_z^*).
\label{ddot_zj}
\end{equation}
%
For the equations for the fields we need to vary the Lagrangian independently with respect to 
$a_z$ and $a_z^*$ and set the variation to zero 
(integration by parts with respect to time  must be done in the action $S=\int dt \LL$ to transform $-a_z^*\delta\dot{a_z}$
to $\dot{a}_z^*\delta a_z$ and similarly for the term $a_z\delta\dot{a}_z^*$)
%
\begin{equation}
\delta\LL = (\ldots)\delta a_z + (\ldots)\delta a_z^* = 0,
\end{equation}
%
which gives two complex conjugate equations:
\begin{equation}
\dot{a}_z = -i\omega_{_{110}}a_z - iec\left(\frac{4\pi}{\sqrt{V}}\right) \sum_{j=1}^{N_p} \dot{z}_j
\label{dot_az}
\end{equation}
%
and an equation for $a_z^*$, which is complex conjugate to \eqref{dot_az}.
%
Defining velocity $v_j = \dot{z}_j$ we can rewrite equations \eqref{ddot_zj} and \eqref{dot_az} 
as first order ODEs:
%
\begin{align}
\dot{z}_j =& v_j \label{dot_zj_110}\, , \\
\dot{v}_j =& \frac{e}{mc}\left(\frac{2}{\sqrt{V}}\right) \left(1-\frac{v_j^2}{c^2} \right)^{3/2} (a_z + a_z^*)
\nonumber\\
 =& \frac{e}{mc}\left(\frac{4}{\sqrt{V}}\right)\left(1-\frac{v_j^2}{c^2} \right)^{3/2}\Re[a_z]
\, ,\label{dot_vj_110} \\
\dot{a}_z =& -i\omega_{_{110}}a_z - iec \left(\frac{4\pi}{\sqrt{V}}\right) \sum_{j=1}^{N_p}v_j\, .
\label{dot_az_110}
\end{align}
%

Note that the system \eqref{dot_zj_110}--\eqref{dot_az_110} includes the electron charge $e$, which is a 
\textit{macrocharge}, i.e., one computational particle may have many (thousands) of real electrons; 
the same is true for the electron mass $m$. However, the mass only enters in the ratio $e/m$ where the
computational weight of $e$ and $m$ cancels out of the particle EOM. The computational weight must
be included in the equation \eqref{dot_az_110} for $a_z$.
Secondly, since the equation for particle $j$ only involves information from the same particle, when
looping over particles as part of the time integration, we may perform the sum over particle velocities needed in
equation \eqref{dot_az_110}, so not much extra computational effort should be involved.
The equation for $a_z$ may be split into real and imaginary parts instead of solving the two complex
conjugate equations for $a_z^*$ and $a_z^*$. 
Equation \eqref{dot_az_110} can be written as:
%
\begin{equation}
\frac{d\Re[{a}_z]}{dt} = \omega_{_{110}}\Im[a_z], \qquad 
\frac{d\Im[{a}_z]}{dt} = -\omega_{_{110}}\Re[a_z] - ec\left(\frac{4\pi}{\sqrt{V}}\right) \sum_{j=1}^{N_p}v_j\, .
\label{dot_az_Re_Im}
\end{equation}
%
The equations are in cgs units.

\newpage
\subsection{Dimensionless variables}
\label{Dimensionless}

We can put all variables in dimensionless form as follows. Time in measured in units of $\omega_0^{-1}$, 
velocity in units of $c$, and vector potential in units of $\gamma_0^{3/2}mc^2/e$ (the extra factor of $\gamma_0^{3/2}$
appears because of the chosen normalization for $\psi_l$, Eq.~\eqref{Eigenfunction_norm}; 
\textcolor{blue}{we may want to change that}), and distance in $c/\omega_0$. 
Here $\omega_0$ is a reference rf frequency (can be chosen as the the frequency of the fundamental cavity mode). 
We can also define a reference wave vector (eigenvalue) $\gamma_0=\omega_0/c$, which means distances
are measured in $\gamma_0^{-1}$. 
The macroparticle weight is $w$. The following two dimensionless parameters appear:
%
\begin{equation}
\xi_0 = \frac{C_{110}}{\gamma_0^{3/2}} = \left(\frac{2} {\gamma_0^{3/2}\sqrt{V}}\right), \qquad 
\eta_0 = \left(\frac{mc^2}{\gamma_0e^2} \right).
\label{dimensionless_parameters}
\end{equation}
%
The values of these parameters depend on the reference wave vector 
$\gamma_0$. For example, for $a=b=7.5\,$cm we have $\gamma_0=0.59$ (choosing the wave number for mode TM$_{110}$).
Choosing the cavity length to be matching the the particle path traveled at $c$ in a wave period, 
we have $d\simeq 5.3\,$cm; then $\xi_0=0.26$, $\eta_0=1.9\times 10^{12}$, and $f_{\rm rf}=\omega_0/(2\pi)=2.826\,$GHz. 
An example for the initial value of (the complex-valued) $a_z$ for electric field amplitude $E_0=50\,$MV/m is 
$a_0 = (E_0\mbox{[V/m]}/(2\gamma_0\xi_0)(e/mc^2)(10^{-4}/3)\simeq 1.93$ (there is a conversion factor $(10^{-4}/3)$ 
of the electric field from SI to cgs and an extra $1/2$ from the relation between 
the real-valued $\mathbf{A}$ and the complex-valued $a_z$).
An example of particle weight (the charge in cgs is in statcoulombs!) for electron current of $1\,$mA 
and $N_p=100$ is $w=2.082\times 10^5$.

We have the following Lagrangian (in units of $mc^2$) with all variables being dimensionless:
%
\begin{align}
\LL =& -\sum_{j=1}^{N_p} w\,\left\{\sqrt{1-\dot{z}_j^2}  + \xi_0
 \dot{z}_j\,2\Re[a_z] \right\} \nonumber\\
& + \eta_0 \left\{ - \frac{1}{2\pi}\omega_{_{110}}^2(a_z a_z^*)
+ \frac{1}{4\pi i} \omega_{_{110}} (a_z\dot{a}_z^* - a_z^*\dot{a}_z ) \right\}\, .
\label{L_dimensionless}
\end{align}
%

The EOM become:
%
\begin{align}
\dot{z}_j =&\, v_j \label{dot_zj_dimensionless}\, , \\
\dot{v}_j =&\, \xi_0 \left(1-v_j^2 \right)^{3/2} (\dot{a}_z + \dot{a}_z^*)
\nonumber\\
 =&\, 2\xi_0\omega_{_{110}}\left(1-v_j^2\right)^{3/2}\Im[a_z]
\, ,\label{dot_vj_dimensionless} \\
\dot{a}_z =& -i\omega_{_{110}}a_z - i\left(\frac{2\pi w\xi_0}{\omega_{_{110}}\eta_0}\right) \sum_{j=1}^{N_p}v_j\, .
\label{dot_az_dimensionless}
\end{align}
%
Equation \eqref{dot_az_dimensionless} is equivalent to two equations for the real and
imaginary parts:
%
\begin{equation}
\frac{d\Re[{a}_z]}{dt} = \omega_{_{110}}\Im[a_z], \qquad 
\frac{d\Im[{a}_z]}{dt} = -\omega_{_{110}}\Re[a_z] - \left(\frac{2\pi w\xi_0}{\omega_{_{110}}\eta_0}\right) \sum_{j=1}^{N_p}v_j\, .
\label{dot_az_Re_Im_dimensionless}
\end{equation}
%

For the above example of initial values, the magnitude of the second term on the rhs of \eqref{dot_az_Re_Im_dimensionless},
relative to the initial value $a_0$, is $\simeq 1.1\times 10^{-6}$; this is reasonable from a numerical viewpoint.

\newpage
\subsection{Benchmarking  in 1D}
\label{Benchmarking}

We compare the numerical result with an analytic calculation of the expected energy gain of an electron 
passing on axis through a cavity with mode TM$_{110}$. The analytic calculation proceeds as follows. 
The electric field on axis in the cavity gap is
%
\begin{equation}
E_z(t) = E_0 \cos(\omega_0 t + \phi_0).
\label{E_gap}
\end{equation}
%
For a particle passing with (almost constant) velocity $c$, we can write
%
\begin{equation}
\omega_0 t = \omega_0 (z/c) = \gamma_0 z.
\end{equation}
%
Then the energy gained by a passing electron is
%
\begin{equation}
\Delta W = \int_{0}^{d}\! dz\, e E_0 \cos(\gamma_0 z+\phi_0) = eE_0d\, \frac{\sin(\gamma_0d/2)}{(\gamma_0d/2)}\cos(\gamma_0d/2+\phi_0)
\label{Delta_W}
\end{equation}
%
The energy gain for the above example ($E_0 = 50\,$MV/m) becomes $\Delta W = 1.6881\,$MeV; the numerical
simulation for a sequence of (initial) values $\beta_0=\{0.9, 0.99, 0.999\}$ gives 
$\Delta W =\{1.6178, 1.6765, 1.6866 \}\,$MeV, which approaches the theoretical value as the velocity of 
the electron approaches $c$.

The table in Fig.~\ref{Fig_Delta_W} compares the analytical and numerical values of $\Delta W$ at $\beta_0=0.999$. 
The plot in Fig.~\ref{Fig_Delta_W} compares the data from the table with formula \eqref{Delta_W}. 
The value of the initial phase $\phi_0$ in formula \eqref{Delta_W} needs to be set to $-\pi/2$ for agreement with 
the numerical initial condition (the electric field is the derivative of the vector potential, i.e., $d(\cos)/dt = -\sin$).
%


\begin{figure}[h]
%%%%%%
\begin{minipage}{3in}
\hspace{0.5cm}
\begin{tabular}{c|c|c}
d[cm] & $\Delta W$[MeV] & $\Delta W$[MeV] \\
 & (analytical)  & (numerical) \\
\hline
2.12 & 0.5832 & 0.5827 \\
4.24 & 1.5269 & 1.5257 \\ 
5.30 & 1.6881 & 1.6866 \\
6.36 & 1.5269 & 1.5251 \\
7.11 & 1.45   & 1.25   \\
8.49 & 0.5832 & 0.5810 \\
10.6 & 0.0    & 0.0    \\
12.7 & 0.5832 & 0.5884 \\
14.9 & 1.5269 & 1.5217 \\
15.9 & 1.6881 & 1.6874 \\
17.0 & 1.5269 & 1.5180 \\
19.1 & 0.5832 & 0.5633 \\
21.2 & 0.0    & 0.0 \\
\hline
\end{tabular}
\end{minipage}
%%%%%%
\hspace{-1cm}
\begin{minipage}{5in}
\includegraphics[width=5in]{Comparison_dW.pdf}
\end{minipage}
%%%%%
\caption{Comparison of analytically obtained energy gain from Eq.~\eqref{Delta_W} with numerical solution of
equations \eqref{dot_zj_dimensionless}--\eqref{dot_az_dimensionless}.}
\label{Fig_Delta_W}
\end{figure}


\newpage
\section{Hamiltonian formulation}
\label{Hamiltonian_formulation}

As noted above, the velocity of an electron doesn't change much in the relevant (wide) energy range 
(10\,MeV--20\,GeV and above). Because of this, round-off errors accumulate in the GeV range of energies. 
At the same time, the momentum of an electron does change significantly as does its energy.  
Therefore, by formulating the equations of motion for the electrons in coordinate-momentum form,
we hope to avoid the numerical round-off errors.

The Hamiltonian of the system of particles and fields can be found from the Lagrangian \eqref{L_dimensionless}
by performing a Legendre transform on \textit{both} the particles and fields. When considering the 
field Lagrangian only, the two variables $a_z$ and $a_z^*$ must be considered as independent, as we discussed
in the derivation of the corresponding field EOM. The generalized momentum (for either particles or fields)
is defined as
%
\begin{equation}
P = \frac{\partial \LL}{\partial \dot{Q}}
\label{Generalized_P}
\end{equation}
%
and the energy of the system, i.e., the Hamiltonian, is defined as
%
\begin{equation}
\HH = P\dot{Q} - \LL
\label{H_definition}
\end{equation}
%
where for many degrees of freedom the first term in \eqref{H_definition} becomes a sum over all
variables (in our case, particles and fields).

\subsection{Hamiltonian}
\label{Hamiltonian}

The corresponding field Hamiltonian is found as follows. The generalized momenta become
%
\begin{equation}
P_a = \frac{\partial \LL}{\partial \dot{a}_z} = - \frac{\eta_0 \omega_{_{110}}}{4\pi i}\, a_z^*\, , \qquad
P_{a^*} = \frac{\partial \LL}{\partial \dot{a}_z} =  \frac{\eta_0 \omega_{_{110}}}{4\pi i}\, a_z\, .
\label{p_a}
\end{equation}
%
Then applying the definition \eqref{H_definition} we obtain
\begin{equation}
\HH_a = \frac{\eta_0\omega_{_{110}}^2}{2\pi}\left( a_z a_z^* \right).
\label{H_a}
\end{equation}
%
For the particle Hamiltonian, by a similar calculation we have a generalized momentum
%
\begin{equation}
P_j = \frac{\partial \LL}{\partial \dot{z}_j} = w\left(\frac{\dot{z}_j}{\sqrt{1-\dot{z}_j^2}} 
- 2\xi_0\Re[a_z] \right)
\label{p_j_definition}
\end{equation}
%
and particle Hamiltonian
%
\begin{equation}
\HH_p = \sum_j\frac{w}{\sqrt{1-\dot{z}_j^2}}
= \sum_j w \sqrt{1 + \left(\frac{P_j}{w} + 2\xi_0\Re[a_z] \right)^2}
\label{H_p}
\end{equation}
%
The full Hamiltonian is the sum of \eqref{H_a} and \eqref{H_p}:
%
\begin{equation}
\HH = \sum_j w \sqrt{1 + \left(\frac{P_j}{w} + 2\xi_0\Re[a_z] \right)^2}
+ \frac{\eta_0\omega_{_{110}}^2}{2\pi}\left( a_z a_z^* \right).
\label{H}
\end{equation}
%

\subsection{Hamiltonian EOM}
\label{Hamiltonian_EOM}


The equations of motion are obtained with the help of the Poisson bracket
defined for any two functions of coordinate and momentum $f(Q,P)$ and $g(Q,P)$ as:
%
\begin{equation}
[f,g] = \frac{\partial f}{\partial Q}\frac{\partial g}{\partial P} 
- \frac{\partial g}{\partial Q}\frac{\partial g}{\partial P}
\label{Poisson_bracket}
\end{equation}
%
with EOM
%
\begin{equation}
\frac{d Q}{d t} = [Q,\HH] = \frac{\partial \HH}{\partial P}\, , \qquad 
\frac{d P}{d t} = [P,\HH] = - \frac{\partial \HH}{\partial Q}\, .
\label{Hamiltonian_EOM}
\end{equation}
%

For our variables \eqref{Poisson_bracket} gives
%
\begin{equation}
[z_i,P_j] = \delta_{ij}\, , \quad [z_i,z_j] = 0\, , \quad [P_i,P_j] = 0\,. % , \quad [a_z, a_z^*] = \frac{1}{i} \, .
\label{Poisson_pq}
\end{equation}
%
The Poisson bracket for the fields is derived in my paper \cite{evstatiev_relativistic_2005}
with the corresponding equation for the fields being identical to \eqref{dot_az_dimensionless}.
The EOM for the electrons become
%
\begin{align}
\dot{z}_j =&\, \frac{(P_j/w) + 2\xi_0\Re[a_z]}{\sqrt{1 + \left(({P_j}/{w}) + 2\xi_0\Re[a_z] \right)^2}}\, ,
\label{dot_zj_Hamiltonian} \\
\dot{P}_j =&\, 0\, , \label{dot_pj_Hamiltonian} \\
\dot{a}_z =& -i\omega_{_{110}}a_z - i\left(\frac{2\pi w\xi_0}{\omega_0\eta_0}\right) \sum_{j=1}^{N_p} \frac{(P_j/w) + 2\xi_0\Re[a_z]}{\sqrt{1 + \left(({P_j}/{w}) + 2\xi_0\Re[a_z] \right)^2}} \, .
\label{dot_az_Hamiltonian}
\end{align}
%
Notice that the time derivative of the generalized momentum equals zero, \eqref{dot_pj_Hamiltonian},
because the mode TM$_{110}$ has no $z$-dependence; the same is true for any mode TM$_{\rm XX0}$.
Another way to say this is that for such modes $z_j$ are cyclic coordinates for the Lagrangian 
\eqref{L_dimensionless}. Similarly to the previous observation, Eq.~\eqref{dot_az_Hamiltonian}
can be replaced by two equations for the real and imaginary parts [cf.~\eqref{dot_az_Re_Im_dimensionless}].


A numerical implementation of Eqs.~\eqref{dot_zj_Hamiltonian}--\eqref{dot_az_Hamiltonian} has shown
that indeed the round-off error has been practically eliminated for the range of energies of 
interest---no noticeable round-off was seen at energies up to 300--400\,GeV.



\section{Transverse dynamics}
\label{Transverse_dynamics}

To study the transverse dynamics of a beam, we add the higher order mode TM$_{220}$ to the fundamental
and expand around $x=a/2$, $y=b/2$:
%
\begin{align}
\psi_{220} &=\, C_{220}\sin\!\left(\frac{2\pi x}{a} \right) \sin\!\left(\frac{2\pi y}{b} \right) %\nonumber \\
\simeq\, \frac{C_{220}4\pi^2}{ab} xy\,.
\label{Mode_220}
\end{align}
%
This should describe a ``dipole'' mode with dependence $x^2-y^2$ [obtainable from \eqref{Mode_220} by a substitution
$x\rightarrow x-y$, $y\rightarrow x+y$]. 
The Lagrangian for the beam in the presence of the two modes, 
TM$_{110}$ and TM$_{220}$, becomes
%
\begin{align}
\LL =& -\sum_{j=1}^{N_p} w\,\left\{\sqrt{1-\dot{x}_j^2-\dot{y}_j^2-\dot{z}_j^2}  
+ 2\xi_{2}\dot{x}_j x_j y_j\Re[a_{_{220},x}] + 2\xi_{2}\dot{y}_jx_j y_j\Re[a_{_{220},y}] 
+ 2\xi_0\dot{z}_j \Re[a_{_{110},z}]
\right\} \nonumber\\
& + \eta_0 \sum_{l={{110},{220}}}\left\{ - \frac{1}{2\pi}\omega_{l}^2(\mathbf{a}_{l}\cdot \mathbf{a}_{l}^*)
+ \frac{1}{4\pi i} \omega_{l} (\mathbf{a}_{l}\cdot \dot{\mathbf{a}}_{l}^* 
- \mathbf{a}_{l}^* \cdot \dot{\mathbf{a}}_{l} ) \right\}
\label{L_transverse}
\end{align}
%
where 
%
\begin{equation}
\xi_2 = \frac{4\pi^2C_{220}}{\gamma_0^{7/2}ab} = \frac{4\pi^2\xi_0}{\gamma_0^2 ab}.
\end{equation}
%

I have assumed that the change in the fundamental mode comes entirely from its longitudinal component $a_z$
and that the change in the ``transverse'' mode TM$_{220}$ comes from its transverse components 
(\textcolor{red}{is this justified?}); i.e., the significant field components are the following three:
$a_{_{110}\,z}$, $a_{_{220}\, x}$, $a_{_{220}\, y}$.

The Hamiltonian for this system of beam and two modes is:
%
\begin{align}
\HH &=\, \sum_{j=1}^{N_p}w\sqrt{1+\left(\frac{P_{jx}}{w}+2\xi_2x_jy_j\Re[a_{_{220\, x}}] \right)^2 
+\left(\frac{P_{jy}}{w}+2\xi_2x_jy_j\Re[a_{_{220\, y}}] \right)^2 
+ \left(\frac{P_{jz}}{w}+2\xi_0\Re[a_{_{110\, z}}] \right)^2  } \nonumber \\
&\phantom{=\,} 
+ \frac{\eta_0\omega_{_{220}}^2}{2\pi}\left(a_{_{220\, x}}a_{_{220\, x}}^* + a_{_{220\, y}}a_{_{220\, y}}^* \right)
+ \frac{\eta_0\omega_{_{110}}^2}{2\pi}\left(a_{_{110\, z}}a_{_{110\, z}}^* \right) 
\label{Hamiltonian_two_mode} \\
&\phantom{=\,}  \equiv w\sum_j\HH_{pj} + \HH_a
\label{Hamiltonian_two_mode_abbrev}
\end{align}
%
where the definitions of $\HH_{pj}$ and $\HH_a$ are obvious.

The equations of motion are obtained with the Poisson brackets \eqref{Poisson_pq}
(the field equations are easiest obtained from the Lagrangian \eqref{L_transverse}):
%
\begin{align}
\dot{x}_j &=\, \frac{(P_{jx}/w)+2\xi_2x_jy_j\Re[a_{_{220\, x}}]}{\HH_{pj}},\quad
\dot{y}_j =\,  \frac{(P_{jy}/w)+2\xi_2x_jy_j\Re[a_{_{220\, y}}]}{\HH_{pj}}, \quad
\dot{z}_j =\,  \frac{(P_{jz}/w)+2\xi_0 \Re[a_{_{110\, z}}]}{\HH_{pj}}\, ,
\label{dot_xyz_two_mode} \\
\dot{P}_{jx} &=\, -\frac{2w\xi_2y_j}{\HH_{pj}}
\left\{ 
\Re[a_{_{220\,x}}]\left( (P_{jx}/w)+2\xi_2x_jy_j\Re[a_{_{220\,x}}]\right)
+ \Re[a_{_{220\,y}}]\left((P_{jy}/w)+2\xi_2x_jy_j\Re[a_{_{220\,y}}]\right)
\right\}, \label{dot_Px_two_mode}\\
\dot{P}_{jy} &=\, -\frac{2w\xi_2x_j}{\HH_{pj}}
\left\{ 
\Re[a_{_{220\,x}}]\left( (P_{jx}/w)+2\xi_2x_jy_j\Re[a_{_{220\,x}}]\right)
+ \Re[a_{_{220\,y}}]\left((P_{jy}/w)+2\xi_2x_jy_j\Re[a_{_{220\,y}}]\right)
\right\}, \label{dot_Py_two_mode}\\
\dot{P}_{jz} &=\, 0 , \label{dot_Pz_two_mode}\\
\dot{a}_{_{110\,z}} &=\, -i\omega_{_{110}}a_{_{110\,z}}
- i\left(\frac{2\pi w\xi_0}{\omega_{_{110}}\eta_0} \right)
\sum_{j=1}^{N_p}\frac{(P_{jz}/w)+2\xi_0\Re[a_{_{110\,z}}]}{\HH_{pj}} ,
\label{dot_az} \\
\dot{a}_{_{220\,x}} &=\, -i\omega_{_{220}}a_{_{220\,x}}
- i\left(\frac{2\pi w\xi_2}{\omega_{_{220}}\eta_0} \right)
\sum_{j=1}^{N_p}\frac{(P_{jx}/w)+2\xi_2x_jy_j\Re[a_{_{220\,x}}]}{\HH_{pj}} ,
\label{dot_ax} \\
\dot{a}_{_{220\,y}} &=\, -i\omega_{_{220}}a_{_{220\,y}}
- i\left(\frac{2\pi w\xi_2}{\omega_{_{220}}\eta_0} \right)
\sum_{j=1}^{N_p}\frac{(P_{jy}/w)+2\xi_2x_jy_j\Re[a_{_{220\,y}}]}{\HH_{pj}} .
\label{dot_ay}
\end{align}



% \vspace{1cm}

% \textcolor{red}{To consider:} 
% i) \textcolor{blue}{Different normalization for the eigenfunctions $\psi_l$.}










\bibliographystyle{unsrt}
\bibliography{Bibliography_BeamEnergyRecoveryModel}

\end{document}